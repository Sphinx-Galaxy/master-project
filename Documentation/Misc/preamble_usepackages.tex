%-------------------------- Documentclass --------------------------
%\documentclass[a4paper,11pt,BCOR5mm,headsepline,liststotoc,bibtotoc]{scrbook} %,smallheadings
\documentclass[%
%draft,     % Entwurfsstadium
draft=false,
final,      % fertiges Dokument
%%%% --- Schriftgröße ---
12pt,
%smallheadings,    % kleine Überschriften
%normalheadings,   % normale Überschriften
%bigheadings,       % große Überschriften
%%%% === Seitengröße ===
% letterpaper,
% legalpaper,
% executivepaper,
a4paper,
% a5paper,
% landscap,
%%%% === Optionen für den Satzspiegel ===
%BCOR5mm,          % Zusaetzlicher Rand auf der Innenseite
%DIV11,            % Seitengroesse (siehe Koma Skript Dokumentation !)
%DIVcalc,         % automatische Berechnung einer guten Zeilenlaenge
%1.1headlines,     % Zeilenanzahl der Kopfzeilen
%headinclude,     % Kopf einbeziehen
%headexclude,      % Kopf nicht einbeziehen
%footinclude,     % Fuss einbeziehen<
%footexclude,      % Fuss nicht einbeziehen
%mpinclude,       % Margin einbeziehen
%mpexclude,        % Margin nicht einbeziehen
pagesize,         % Schreibt die Papiergroesse in die Datei.
% Wichtig fuer Konvertierungen
%%%% === Layout ===
%oneside,         % einseitiges Layout
twoside,          % Seitenraender für zweiseitiges Layout
%onecolumn,        % Einspaltig
%twocolumn,       % Zweispaltig
%openany,         % Kapitel beginnen auf jeder Seite
openright,        % Kapitel beginnen immer auf der rechten Seite
% (macht nur bei 'twoside' Sinn)
%cleardoubleplain,    % leere, linke Seite mit Seitenstil 'plain' 
cleardoublepage=empty,% leere, linke Seite mit Seitenstil 'empty'
titlepage,        % Titel als einzelne Seite ('titlepage' Umgebung)
%notitlepage,     % Titel in Seite integriert
captions=tableheading,
%%%% --- Absatzeinzug ---
%                 % Absatzabstand: Einzeilig,
%parskip,         % Freiraum in letzter Zeile: 1em
%parskip*,        % Freiraum in letzter Zeile: Viertel einer Zeile
%parskip+,        % Freiraum in letzter Zeile: Drittel einer Zeile
%parskip-,        % Freiraum in letzter Zeile: keine Vorkehrungen
%                 % Absatzabstand: Halbzeilig
%halfparskip,     % Freiraum in letzter Zeile: 1em
%halfparskip*,    % Freiraum in letzter Zeile: Viertel einer Zeile
%halfparskip+,    % Freiraum in letzter Zeile: Drittel einer Zeile
%halfparskip,     % Freiraum in letzter Zeile: keine Vorkehrungen
%                 % Absatzabstand: keiner
parskip=false,    % Eingerückt (Standard)
%%%% --- Kolumnentitel ---
headsepline,      % Linie unter Kopfzeile
%headnosepline,   % keine Linie unter Kolumnentitel
footsepline,      % Linie über Fusszeile
%footnosepline,   % keine Linie unter Fussnote
plainfootsepline, % Linie über Fußzeile auf "`leeren Seiten"'
%%%% --- Kapitel ---
chapterprefix,   % Ausgabe von 'Kapitel:'
%nochapterprefix,  % keine Ausgabe von 'Kapitel:'
%%%% === Verzeichnisse (TOC, LOF, LOT, BIB) ===
%listof=totoc,      % Tabellen & Abbildungsverzeichnis ins Inhaltsverzeichnis
%listof=notoc,	  % Tabellen & Abbildungsverzeichnis ins nicht ins Inhaltsverzeichnis
%idxtotoc,        % Index ins TOC
bibliography=totoc, % Bibliographie ins TOC
%bibtotocnumbered, % Bibliographie im TOC nummeriert
%liststotocnumbered, % Alle Verzeichnisse im TOC nummeriert      
toc=graduated,    % eingereuckte Gliederung
%tocleft,         % Tabellenartige TOC
%listof=graduated,      % eingereuckte LOT, LOF
%listsleft,       % Tabellenartige LOT, LOF
%pointednumbers,  % Überschriftnummerierung mit Punkt, siehe DUDEN !
%numbers=noenddot, % Überschriftnummerierung ohne Punkt, siehe DUDEN !
%openbib,         % alternative Formatierung des Literaturverzeichnisses
%%%% === Matheformeln ===
%leqno,           % Formelnummern links
%fleqn,            % Formeln werden linksbuendig angezeigt
]{scrbook}%     Klassen: scrartcl, scrreprt, scrbook

% Kodierung und Lokalisierung
\usepackage[T1]{fontenc}     	% LaTeX-interne Schriftkodierung
\usepackage[utf8]{inputenc}  	% UTF8! Test: äöüÄÖÜß (Umlaute müssten richtig dargestellt werden)
\usepackage[british]{babel}  	% Deutsche Titel, Silbentrennung, etc.
\usepackage{lmodern}			% Skalierbare Fonts


%Text
\usepackage[onehalfspacing]{setspace}	%Zeilenabstand
\usepackage[a4paper,left=1in, right=1in, bottom=1.5in]{geometry}	%feste Definition der Seitenränder
\usepackage[svgnames,table, hyperref]{xcolor}
\usepackage[hidelinks]{hyperref}			% interne Verlinkungen in klickbare Verweise
\usepackage{relsize}			% Schriftgröße verändern
\clubpenalty=10000 				% Schusterjungen
\widowpenalty=10000 			% Hurenkinder
\displaywidowpenalty=10000 		% Für Math-Modus
\setlength\parindent{0pt} 		% keine Einrückung beim Absatz
\usepackage{microtype}			% Mikrotypographie

%Seitenlayout für die Seiten auf denen ein neues Kapitel beginnt
\setkomafont{chapterprefix}{\Large}
\setkomafont{chapter}{\raggedright\Huge}
\renewcommand*{\chapterformat}{%
	\raggedleft%
	\MakeUppercase{\chapapp}{\Huge\nobreakspace\thechapter}
	\vskip .2ex
	\raisebox{\ht\strutbox}[.5pt][0pt]{\rule{\textwidth}{.5pt}}%
	\vskip-1\baselineskip
}
\renewcommand*{\chapterheadendvskip}{%
\vspace{-1\baselineskip}\noindent{\setlength{\parskip}{0pt}\Huge\hrulefill\par}%
\vspace{\baselineskip}
}

%Kopf-/Fußzeilen
\usepackage{scrlayer-scrpage}
\pagestyle{scrheadings} 

% Abstände
\setlength{\parskip}{1ex plus 0.5ex minus 0.2ex}
\setlength{\parindent}{0mm}
\setlength{\itemsep}{0ex plus 0.2ex}

% Abkuerzungsverzeichnis: (list of acronyms)
\usepackage[printonlyused]{acronym} %only print acronyms which are used in the text

%Mathe und Co
\usepackage{dsfont}					% Einheitsmatrix
\usepackage{amsfonts}   			% Matheumgebungen
\usepackage{amssymb}
\usepackage{amsmath}  				
\usepackage[right]{eurosym} 		% Euro-Zeichen
\usepackage{textcomp} 				% Zusätzliche Symbolzeichen

%Grafiksatz
\usepackage{color}					% Farben
\usepackage{xcolor}					% Farben
\usepackage{graphicx}				% Bildsatz
\usepackage{pdfpages} 				% Kann PDF-Dateien einbinden
\usepackage[bf]{caption}			% fette Abbildungbeschriftung
\usepackage{subcaption}				% Unterbilder (Fig. 1a)
\usepackage{acronym}				% Abkürzungsverzeichnis
\usepackage[some]{background}		% Hintergründe (Titelseite)


\usepackage{lscape}
% Literaturverweise mit (Autor Jahr) nach DIN
\usepackage[numbers,square]{natbib} 
%\bibpunct{[}{]}{;}{x}{,}{,}


% Quellcode
\usepackage{listings}				% Source-Code einbinden 


%Einheiten
\usepackage[]{units}
\usepackage{siunitx}
\DeclareSIUnit\byte{Byte}

% Tikz/Pgfplots 
\usepackage{tikz}
\usepackage{pgfplots,ifthen}
\pgfplotsset{compat=1.17}
\pgfkeys{/pgf/number format/.cd ,use comma ,set thousands separator={ }}
\usepgfplotslibrary{groupplots}
\usepackage{tikzscale}				% Skalierung


% Tabellen
\usepackage{booktabs} 				% Besonders schöne Tabellen (Bottomrule...)
\usepackage{tabularx}				% Erweiterungen
\usepackage{longtable}				% Tabelle mit Seitenumbruch
\usepackage{supertabular}			% ""
\usepackage{diagbox}
\usepackage{multirow}				% mehrzeilige Tabellen
\usepackage{array}


\usepackage{bigstrut} % fuer Tabellen (Zeilen mit kleinem Abstand und nicht gequetscht!)

\usepackage{rotating} % hier: um eine große Tabelle zu drehen
%	% BEISPIEL
%	\begin{sidewaystable}
%  	 \begin{tabular}{@{}*{4}{l}@{}}
%  	       Nominativ & Genetiv & Dativ & Akkusativ \\
%  	       die Frau & der Frau & der Frau & die Frau
%  	 \end{tabular}
%	\end{sidewaystable} 

%\renewcommand\tabcolsep{3pt} % die Zahl (hier: 3) gibt den Abstand zwischen den Spalten für alle Tabellen an (default ist: 6pt)
%--------------------------------------------


%Beautiful lstlistning
\DeclareCaptionFont{white}{\color{white}}
\DeclareCaptionFormat{listing}{%
 \parbox{\textwidth}{\colorbox{gray}{\parbox{0.95\textwidth}{#1#2#3}}\vskip-4pt}}
\captionsetup[lstlisting]{format=listing,labelfont=white,textfont=white}
\lstset{frame=lrb,xleftmargin=\fboxsep,xrightmargin=0.045\textwidth}