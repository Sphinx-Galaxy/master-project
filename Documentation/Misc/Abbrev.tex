\chapter*{Abkürzungsverzeichnis}\label{Kapitel:abbrev} % \ref{Kapitel:abbrev} 


% Redefine font for acronyms
\renewcommand*{\acsfont}[1]{\normalfont{\normalsize{#1}}} %default is BOLD for short versions - Problem when symbols are used


% Add acronyms here

\begin{acronym}[LONGESTACRO] %Zur Definition des Abstandes im Verzeichnis	
	% \setlength{\itemsep}{-\parsep} % engerer Abstand zwischen den Zeilen
		
	% Abkuerzungsdefinition: \acro{Verweis}[Abkuerzung]{ausgeschriebene Abkuerzung}	
	% Aufruf im Text:	
	%    \ac{Verweis}  = normaler Aufruf ODER	
	%    \acs{Verweis} = short version ODER	
	%    \acl{}  = long version (without acronym) ODER	
	%    \acf{}  = full version ODER	
	%    \acp{}  = plural	
	%    \acsp{} = short + plural	
		
	% Es gibt auch die M\"oglichkeit gezielt Einfluss auf die plurale Form zu nehmen:	
	% \acroplural{<acronym>}[<short plural>]{<long plural>}	
	% \newacroplural{<acronym>}[<short plural>]{<long plural>}	
	% \acrodefplural{<acronym>}[<short plural>]{<long plural>}
			
	%Sortierung entspricht der späteren Anzeigereihenfolge. Ggf. also alphabetisch sortieren (Funktionalität im Editor nutzen!?)	
	% Befehle mit „\acroextra{}“ werden nur im Verzeichnis ausgegeben.
			
	% Input of all acronyms starts here:	
	%\acro{<command>}[<short>]{<long>} note: command is not able to consist of 'special characters' cmp. as a label
	
	\acro{AD}    [A/D]   {Analog to Digital}	
	\acro{ASIC}  [ASIC]  {Application Specific Integrated Circuit\acroextra{ (\textit{anwendungsspezifische integrierte Schaltung})}}
	
	
	
\end{acronym}


